\documentclass[11pt]{article}
% Language setting
% Replace `english' with e.g. `spanish' to change the document language

\usepackage{format}
\addbibresource{references.bib}
\begin{document}

\author[1, 2]{Davidson Lova Razafindrakoto}
% \author[1]{Alain Celisse}
% \author[2]{Jérôme Lacaille}
\affil[1]{Laboratoire SAMM, Université Paris 1 Panthéon-Sorbonne, Paris, France.
}
\affil[2]{Safran Aircraft Engines, Paris, France.
}

\title{
    Instant-by-instant and simultaneous confidence prediction bands
}

\date{davidson-lova.razafindrakoto@safrangroup.com}

%%%%%%%%%%
\maketitle

%%%%%
% \begin{abstract}
%     instant-by-instant and simultaneous confidence prediction bands.
% \end{abstract}

% Keywords: Conformal prediction, nested prediction sets, point-wise confidence bands.

% \section{Introduction}
% \label{sec.intro}

% [Here we are going to insert the reason why we do all of these]

% \citet{izbicki2019flexible} proposed two methods for building prediction bands:
% CDist-split and CD-split based on split conformal prediction \citep{vovk2005algorithmic}.
% % CDist-split uses a transformation of an estimate of conditional cumulative density function as a conformity measure,
% % whereas CD-split uses an estimate of the conditional probability density function as a conformity measure.
% The prediction sets forming CDist-split bands are intervals whereas
% those forming CD-split ones are (estimated) density level sets which can take any form depending the shape of the conditional probability density function estimate
% (for instance a union of two intervals if the density is bimodal).

% \citet{izbicki2022cd} proposed yet another method for building prediction bands : HPD-split.
% % HPD-split uses an (estimated) probability measure of an (estimated) conditional density level set
% % where the level is an estimate of conditional probability density.
% The prediction sets forming HPD-split bands are also (estimated) density level sets.
% What differs from CD-split is that the chosen level varies with the input value, whereas for CD-split, the level is the same for every input value.
% % These three prediction bands (CDist-split, CD-split, HPD-split) mentioned here are indexed by the input value, and
% % they ensure control of the coverage at each input value individually.
% In our context, our instant-by-instant prediction bands are index by the instant.
% Since for each instant, the probability distribution is supported on a discrete finite set,
% full conformal (no splitting) versions of CDist-split and CD-split are proposed.


% \citet{diquigiovanni2021importance} proposed a method for building prediction bands for functional data
% based on split conformal prediction \citep{vovk2005algorithmic}.
% % They chose the supremum of the residuals (and other refinements) across all instants as non-conformity score.
% This prediction band ensure control of coverage of the (random) function at every instant simultaneously.
% In contrast to their functions which are indexed by a continuous interval of instants,
% our curves are indexed by a finite discrete set of instants.
% \todo{Davidson: This is a little weak}
% In contrast to their bands, our $\gamma$-simultaneous bounds
% ensure control of coverage of the (random) curve only for a specified (supposed to be large) proportion of instants.


% Section \ref{sec.ibib} discusses instant-by-instant prediction bands
% based on full conformal prediction \citep{vovk2005algorithmic} version of CD-split, CDist-split in our context,
% along with one based on split conformal prediction of CDist-split.

% Section \ref{sec.simb} discusses $\gamma$-simultaneous prediction bands
% based on the previously mentioned instant-by-instant prediction bands.

% \section{Instant-by-Instant prediction band}
% \label{sec.ibib}



% \section{Simultaneous prediction band}
% \label{sec.simb}



% \newpage
\section*{Appendix}
\addcontentsline{toc}{section}{Appendix}
\renewcommand{\thesubsection}{\Alph{subsection}}

\begin{definition}
    Let $K$ the size of the fleet and $T$ the size of the time grid be non-zero integers.
    Let a cumulative event number evolution curve $\xi = \paren{\xi_1, \ldots, \xi_T}$ be a random vector where
    for every  $\tau \in \brac{1, \ldots, T}$, $\xi_{\tau}$ is a $\brac{1, \ldots, K}$-valued random variable.
    Let a band $B = (B_{1}, \ldots, B_{T})$ be such that for every $\tau \in \brac{1, \ldots, K}, \hat{B}_{\alpha; \tau} \subseteq \brac{1, \ldots, K}$.
\end{definition}

\begin{definition}[Instant-by-instant coverage]
    Let $\alpha \in (0, 1)$.
    A band $B$ controls the instant-by-instant coverage of the curve $\xi$ at a control level $\alpha$
    if
    \begin{align*}
        \forall \tau \in \brac{1, \ldots, T},
        \quad
        \mathbb{P}\croch{\xi_\tau \in \hat{B}_{\alpha; \tau}} \geq 1 - \alpha.
    \end{align*}
\end{definition}


\begin{definition}[$\gamma$-Simultaneous coverage]
    Let $\alpha \in (0, 1)$ and $\gamma \in \croch{0, 1}$.
    A band $B$ controls the $\gamma$-simultaneous coverage of the curve $\xi$ at a control level $\alpha$
    if
    \begin{align*}
        \mathbb{P}\croch{
            \frac{\mathrm{Card} \paren{
                \brac{
                    \tau \in \brac{1, \ldots, T}:
                    \xi_\tau \in \hat{B}_{\alpha; \tau}
                }
            }}{T}
            \geq 1 - \gamma
        } \geq 1 - \alpha.
    \end{align*}
\end{definition}


\begin{proposition}[Instant-by-instant prediction band, MD-full]
\label{prop.ibib.md.full}
    Let $\xi_1, \ldots, \xi_n$ be $n$ independent copies of $\xi$.
    For any control level $\alpha \in \croch{0, 1}$,
    the band $\hat{B}^{\mathrm{MD-full}}_{\alpha}$ defined as, for every $\tau \in \brac{1, \ldots, T}$
    \begin{align}
    \label{eq.ibib.md.full}
        \hat{B}_{\alpha; \tau}^{\mathrm{MD-full}} := \brac{
            k \in \brac{1, \ldots, K} :
            \hat{p}_{\tau}(k) \geq \hat{\ell}_{\alpha; \tau}  - \frac{1}{n}
        }
    \end{align}
    ensures control of the instant-by-instant coverage of $\xi$ at a control level $\alpha$,
    where the marginal empirical probability density function $\hat{p}_{\tau}$ is defined in Equation (Eq.~\eqref{eq.epmf.full})
    and the threshold $\hat{\ell}_{\alpha; \tau}$ in Equation (Eq.~\eqref{eq.threshold.md.full}).
\end{proposition}
\begin{proof}[Proof of Proposition \ref{prop.ibib.md.full}]
    Let $\alpha \in \croch{0, 1}$ and $\tau \in \brac{1, \ldots, T}$.
    Following \citet{lei2013distribution},
    a full conformal prediction $\hat{C}_{\alpha; \tau}$ region is built form
    the observations $\xi_{1; \tau}, \ldots, \xi_{n; \tau}$ for $\xi_{\tau}$
    as follows
    \begin{align*}
        \hat{C}_{\alpha; \tau} =
        \brac{k \in \brac{1, \ldots, K}
        : \hat{\pi}_{\tau} (k) > \alpha
    }
    \end{align*}
    %
    %
    %
    where for every $k \in \brac{1, \ldots, K}$
    \begin{align*}
        \hat{\pi}_{\tau} (k)
        := \frac{
            1
            + \sum_{i = 1}^{n}
            \mathbbm{1} \brac{
                \hat{p}_{\tau}^{k}\paren{\xi_{i; \tau}}
                \leq \hat{p}^{k}_{\tau}(k)
            }
        }{n+1}
    \end{align*}
    %
    %
    %
    where for every $l \in \brac{1, \ldots, K}$
    \begin{align*}
        \hat{p}_{\tau}^{k}(l) = \frac{n}{n+1} \hat{p}_{\tau}(l) + \frac{1}{n+1} \mathbbm{1} \brac{k = l},
        \end{align*}
    and the empirical marginal probability density function is given by
    \begin{align}
        \label{eq.epmf.full}
        \hat{p}_{\tau}(l) = \frac{1}{n} \sum_{i=1}^{n} \mathbbm{1} \brac{\xi_{i; \tau} = l}
    \end{align}
    %
    %
    %
    Since $\xi_{1; \tau}, \ldots, \xi_{n; \tau}$ are i.i.d. copies of $\xi_{\tau}$,
    $\hat{C}_{\alpha; \tau}$ is a confidence prediction region \citep{vovk2005algorithmic}
    \begin{align*}
        \mathbb{P} \croch{
            \xi_{\tau} \in \hat{C}_{\alpha; \tau}
        } \geq 1 - \alpha.
    \end{align*}
    %
    %
    %
    Let us now provide an explicit expression of this region.
    \begin{align*}
        k \in \hat{C}_{\alpha; \tau}
        &
        \Longleftrightarrow \hat{\pi}_{\tau}\paren{k} > \alpha
        \\
        &
        \Longleftrightarrow \frac{
            1
            + \sum_{i = 1}^{n}
            \mathbbm{1} \brac{
                \hat{p}_{\tau}^{k}\paren{\xi_{i; \tau}}
                \leq \hat{p}^{k}_{\tau}(k)
            }
        }{n+1} > \alpha
        \\
        &
        \Longleftrightarrow 
        \frac{1}{n}
        \sum_{i = 1}^{n}
        \mathbbm{1} \brac{
            \hat{p}_{\tau}^{k}\paren{\xi_{i; \tau}}
            > \hat{p}^{k}_{\tau}(k)
        }
        < \paren{1 - \alpha} \paren{1 + \frac{1}{n}}.
    \end{align*}
    %
    %
    %
    Let us consider the term on the l.h.s.
    %
    %
    %
    \begin{align*}
        &
        \frac{1}{n}
        \sum_{i = 1}^{n}
        \mathbbm{1} \brac{
            \hat{p}_{\tau}^{k}\paren{\xi_{i; \tau}}
            > \hat{p}^{k}_{\tau}(k)
        }
        \\
        &
        =
        \frac{1}{n}
        \sum_{i = 1}^{n}
        \sum_{l = 1}^{K}
        \mathbbm{1} \brac{l = \xi_{i; \tau}}
        \mathbbm{1} \brac{
            \hat{p}_{\tau}^{k}\paren{\xi_{i; \tau}}
            > \hat{p}^{k}_{\tau}(k)
        }
        \\
        &
        =
        \sum_{l = 1}^{K}
        \hat{p}_{\tau}(l)
        \mathbbm{1} \brac{
            \hat{p}_{\tau}^{k}(l)
            > \hat{p}^{k}_{\tau}(k)
        }
        \\
        &
        =
        \sum_{l = 1}^{K}
        \hat{p}_{\tau}(l)
        \mathbbm{1} \brac{
            \frac{n}{n+1} \hat{p}_{\tau}(l)
            + \frac{1}{n+1} \mathbbm{1} \brac{k = l}
            >
            \frac{n}{n+1} \hat{p}_{\tau}(k)
            + \frac{1}{n+1}
        }
        \\
        &
        =
        \sum_{l = 1}^{K}
        \hat{p}_{\tau}(l)
        \mathbbm{1} \brac{
            \hat{p}_{\tau}(l)
            + \frac{1}{n} \mathbbm{1} \brac{k = l}
            >
            \hat{p}_{\tau}(k)
            + \frac{1}{n}
        }
        \\
        &
        =
        \hat{p}_{\tau}(k)
        \mathbbm{1} \brac{
            \hat{p}_{\tau}(k)
            + \frac{1}{n}
            >
            \hat{p}_{\tau}(k)
            + \frac{1}{n}
        }
        +
        \sum_{l = 1, l \neq k}^{K}
        \hat{p}_{\tau}(l)
        \mathbbm{1} \brac{
            \hat{p}_{\tau}(l)
            >
            \hat{p}_{\tau}(k)
            + \frac{1}{n}
        }
        \\
        &
        =
        \hat{p}_{\tau}(k)
        \mathbbm{1} \brac{
            \hat{p}_{\tau}(k)
            >
            \hat{p}_{\tau}(k)
            + \frac{1}{n}
        }
        +
        \sum_{l = 1, l \neq k}^{K}
        \hat{p}_{\tau}(l)
        \mathbbm{1} \brac{
            \hat{p}_{\tau}(l)
            >
            \hat{p}_{\tau}(k)
            + \frac{1}{n}
        }
        \\
        &
        =
        \sum_{l = 1}^{K}
        \hat{p}_{\tau}(l)
        \mathbbm{1} \brac{
            \hat{p}_{\tau}(l)
            >
            \hat{p}_{\tau}(k)
            + \frac{1}{n}
        }.
    \end{align*}
    It follows that
    \begin{align*}
        k \in \hat{C}_{\alpha; \tau}
        &
        \Longleftrightarrow
        \sum_{l = 1}^{K}
        \hat{p}_{\tau}(l)
        \mathbbm{1} \brac{
            \hat{p}_{\tau}(l)
            >
            \hat{p}_{\tau}(k)
            + \frac{1}{n}
        } < \paren{1 - \alpha} \paren{1 + \frac{1}{n}}
        \\
        &
        \Longleftrightarrow
        \hat{p}_{\tau}(k) \geq \hat{\ell}_{\alpha; \tau},
    \end{align*}
    where the threshold $\hat{\ell}_{\alpha; \tau}$ is defined as
    \begin{align}
        \label{eq.threshold.md.full}
        \hat{\ell}_{\alpha; \tau}
        := \inf
        \brac{t \in [0, 1]: 
        \sum_{l = 1}^{K}
        \hat{p}_{\tau}(l)
        \mathbbm{1} \brac{
            \hat{p}_{\tau}(l)
            >
            t
        } < \paren{1 - \alpha} \paren{1 + \frac{1}{n}}}.
    \end{align}
    %
    %
    %
    \textbf{How to compute $\hat{\ell}_{\alpha; \tau}$}.
    The function $G : [0, 1] \to [0, 1]$, $t \mapsto \hat{G}_{\tau}(t) := \sum_{l = 1}^{K}
        \hat{p}_{\tau}(l)
        \mathbbm{1} \brac{
            \hat{p}_{\tau}(l)
            >
            t
    }$
    is decreasing piece-wise constant right-continuous function
    which changes value at $\hat{p}_{\tau}(1), \ldots, \hat{p}_{\tau}(K)$.
    Let $x_1 > \ldots > x_{\kappa}$
    be the distinct values among $\hat{p}_{\tau}(1), \ldots, \hat{p}_{\tau}(K)$
    sorted from largest to smallest,
    and for all $j \in \brac{1, \ldots, \kappa}$,
    $\rho_j := \sum_{l = 1}^{K} \hat{p}_{\tau}(k) \mathbbm{1} \brac{\hat{p}_{\tau}(k) = x_j}$.
    Consider the cumulative sum of $\rho_1, \ldots, \rho_{\kappa}$, $s_{1}, \ldots s_{\kappa}$.
    It follows that $G(x_1) = 0$ and for every $j \in \brac{2, \ldots, \kappa}$
    \begin{align*}
        G(x_j)
        &
        = \sum_{l = 1}^{K}
            \hat{p}_{\tau}(l)
            \mathbbm{1} \brac{
                \hat{p}_{\tau}(l)
            >
            x_j
        }
        = \sum_{l = 1}^{K}
            \sum_{i = 1}^{\kappa}
            \mathbbm{1}
            \brac{x_i = \hat{p}_{\tau}(l)}
            \hat{p}_{\tau}(l)
            \mathbbm{1} \brac{
                \hat{p}_{\tau}(l)
            >
            x_j
        }
        \\
        &
        =
        \sum_{i = 1}^{\kappa}
            \rho_i
            \mathbbm{1} \brac{
                x_i > x_j
        }
        =
        \sum_{i = 1}^{j-1}
        \rho_i = s_{j-1}.
    \end{align*}
    It follows that, one only has to check for finite number of values
    \begin{align*}
        \hat{\ell}_{\alpha; \tau}
        &
        = \inf
        \brac{
            t \in \croch{0, 1}
            : \hat{G}_{\tau}(t) < \paren{1 - \alpha}\paren{1 + \frac{1}{n}} 
        }
        \\
        &
        =
        \min
        \brac{
            x_i
            : 
            i \in \brac{1, \ldots, \kappa},
            s_{i-1} < \paren{1 - \alpha}\paren{1 + \frac{1}{n}}
        }.
    \end{align*}
\end{proof}


\begin{proposition}[Instant-by-instant prediction upper band, MDist-full]
\label{prop.ibib.mdist.up.full}
Let $\xi_1, \ldots, \xi_n$ be $n$ independent copies of $\xi$.
%
%
%
For any control level $\alpha \in \croch{0, 1}$,
the band $\hat{B}^{\mathrm{MDist-full, up}}_{\alpha}$ defined as,
for every $\tau \in \brac{1, \ldots, T}$
\begin{align}
\label{eq.ibib.mdist.up.full}
    \hat{B}_{\alpha; \tau}^{\mathrm{MDist-full, up}} := \brac{
        k \in \brac{1, \ldots, K} :
        k \leq \hat{Q}_{\tau}^{\mathrm{up}} \paren{\paren{1 - \alpha} \paren{1 + \frac{1}{n}}} + 1
    }
\end{align}
ensures control of the instant-by-instant coverage of $\xi$ at a control level $\alpha$,
the empirical marginal upper quantile function $\hat{Q}^{\mathrm{up}}_{\tau}$ is defined in Equation (Eq.~\eqref{eq.eqt.up.full}).
\end{proposition}
%
%
%
\begin{proof}[Proof of Proposition \ref{prop.ibib.mdist.up.full}]
Let $\alpha \in \croch{0, 1}$ and $\tau \in \brac{1, \ldots, T}$.
A full conformal prediction region $\hat{C}_{\alpha; \tau}$ can be defined as
\begin{align*}
    \hat{C}_{\alpha; \tau}
    := \brac{
        k \in \brac{1, \ldots, K}:
        \hat{\pi}\paren{k} > \alpha
    },
\end{align*}
where for every $k \in \brac{1, \ldots, K}$
\begin{align*}
    \hat{\pi}\paren{k} :=
    \frac{1 
        + \sum_{i=1}^{n}
        \mathbbm{1}
        \brac{
            \hat{F}_{\tau}^{k} \paren{\xi_{i; \tau}}
            \geq \hat{F}_{\tau}^{k} \paren{k}
        }
    }{n+1},
\end{align*}
and for every $l \in \brac{1, \ldots, K}$
\begin{align*}
    \hat{F}_{\tau}^{k} \paren{l}
    := \frac{n}{n+1} \hat{F}_{\tau}\paren{l}
    + \frac{1}{n+1} \mathbbm{1} \brac{k \leq l},
\end{align*}
where the empirical marginal cumulative density function $\hat{F}_{\tau}$ is defined as
\begin{align}
    \label{eq.ecdf.full}
    \hat{F}_{\tau} \paren{l}
    := \frac{1}{n} \sum_{i=1}^{n} \mathbbm{1} \brac{\xi_{i; \tau} \leq l}.
\end{align}
%
%
%
Since $\xi_{1; \tau}, \ldots, \xi_{n; \tau}$ are i.i.d. copies of $\xi_{\tau}$,
$\hat{C}_{\alpha; \tau}$ is a confidence prediction region \citep{vovk2005algorithmic}
\begin{align*}
    \mathbb{P} \croch{
        \xi_{\tau} \in \hat{C}_{\alpha; \tau}
    } \geq 1 - \alpha.
\end{align*}
%
%
%
Let us now provide an explicit expression of $\hat{C}_{\alpha; \tau}$.
\begin{align*}
    k \in \hat{C}_{\alpha; \tau}
    &
    \Longleftrightarrow
    \hat{\pi}_{\tau}\paren {k} > \alpha
    \\
    &\Longleftrightarrow
    \frac{1 
        + \sum_{i=1}^{n}
        \mathbbm{1}
        \brac{
            \hat{F}_{\tau}^{k} \paren{\xi_{i; \tau}}
            \geq \hat{F}_{\tau}^{k} \paren{k}
        }
    }{n+1} > \alpha
    \\
    &\Longleftrightarrow
    \frac{1}{n}
    \sum_{i=1}^{n}
    \mathbbm{1}
    \brac{
        \hat{F}_{\tau}^{k} \paren{\xi_{i; \tau}}
        < \hat{F}_{\tau}^{k} \paren{k}
    }
    <
    \paren{1 - \alpha} \paren{1 + \frac{1}{n}}.
    % \alpha \frac{n+1}{n} - \frac{1}{n}
\end{align*}
%
%
%
Let us focus on the l.h.s. term. Let us define the function $\hat{G}_{\tau} : \brac{1, \ldots, K} \to \croch{0, 1}$
such that for every $k$
\begin{align*}
    \hat{G}_{\tau}(k)
    &
    :=
    \frac{1}{n}
    \sum_{i=1}^{n}
    \mathbbm{1}
    \brac{
        \hat{F}_{\tau}^{k} \paren{\xi_{i; \tau}}
        < \hat{F}_{\tau}^{k} \paren{k}
    }
    \\
    &
    =
    \frac{1}{n}
    \sum_{i=1}^{n}
    \sum_{l=1}^{K}
    \mathbbm{1}
    \brac{l = \xi_{i; \tau}}
    \mathbbm{1}
    \brac{
        \hat{F}_{\tau}^{k} \paren{\xi_{i; \tau}}
        < \hat{F}_{\tau}^{k} \paren{k}
    }
    \\
    &
    =
    \sum_{l=1}^{K}
    \hat{p}_{\tau}\paren{l}
    \mathbbm{1}
    \brac{
        \hat{F}_{\tau}^{k} \paren{l}
        < \hat{F}_{\tau}^{k} \paren{k}
    }
    \\
    &
    =
    \sum_{l=1}^{K}
    \hat{p}_{\tau}\paren{l}
    \mathbbm{1}
    \brac{
        \frac{n}{n+1}
        \hat{F}_{\tau}\paren{l}
        + \frac{1}{n+1} \mathbbm{1} \brac{k \leq l}
        <
        \frac{n}{n+1} 
        \hat{F}_{\tau}\paren{k}
        + \frac{1}{n+1}
    }
    \\
    &
    =
    \sum_{l=1}^{K}
    \hat{p}_{\tau}\paren{l}
    \mathbbm{1}
    \brac{
        \hat{F}_{\tau}\paren{l}
        + \frac{1}{n} \mathbbm{1} \brac{k \leq l}
        < \hat{F}_{\tau}\paren{k}
        + \frac{1}{n}
    }
    \\
    &
    =
    \sum_{l=1, l < k}^{K}
    \hat{p}_{\tau}\paren{l}
    \mathbbm{1}
    \brac{
        \hat{F}_{\tau}\paren{l}
        < \hat{F}_{\tau}\paren{k}
        + \frac{1}{n}
    }
    +
    \sum_{l=1, l \geq  k}^{K}
    \hat{p}_{\tau}\paren{l}
    \mathbbm{1}
    \brac{
        \hat{F}_{\tau}\paren{l}
        < \hat{F}_{\tau}\paren{k}
    }.
\end{align*}
%
%
%
If $1 \leq l < k$ then $\hat{F}_{\tau}\paren{l} \leq \hat{F}_{\tau}\paren{k}<\hat{F}_{\tau}\paren{k} + \frac{1}{n}$,
and if $K \geq l \geq k$ then $\hat{F}_{\tau}\paren{l} \geq \hat{F}_{\tau}\paren{k}$.
The above equation can be rewritten as
\begin{align*}
    \hat{G}_{\tau}\paren{k} 
    = \sum_{l=1, l < k}^{K}
    \hat{p}_{\tau} \paren{l}
    = \hat{F}_{\tau} \paren{k-1}.
\end{align*}
%
%
%
Going back to trying to find an expression for $\hat{C}_{\alpha; \tau}$
\begin{align*}
    k \in \hat{C}_{\alpha; \tau}
    &
    \Longleftrightarrow
    \hat{F}_{\tau} \paren{k-1}
    <
    \paren{1 - \alpha} \paren{1 + \frac{1}{n}}
    \\
    &
    \Longleftrightarrow
    k-1
    \leq
    \hat{Q}_{\tau} \paren{\paren{1 - \alpha} \paren{1 + \frac{1}{n}}}
    \\
    &
    \Longleftrightarrow
    k
    \leq
    \hat{Q}_{\tau} \paren{\paren{1 - \alpha} \paren{1 + \frac{1}{n}}} + 1,
\end{align*}
where the empirical marginal upper quantile function is defined as, for every level $a \in \croch{0, 1}$,
\begin{align}
    \label{eq.eqt.up.full}
    \hat{Q}_{\tau}^{\mathrm{up}} \paren{a}
    := \max \brac{
        k \in \brac{1, \ldots, K} :
        \hat{F}_{\tau} < a.
    }
\end{align}
the empirical marginal cumulative distribution function $\hat{F}_{\tau}$ in Equation (Eq.~\eqref{eq.ecdf.full}).
\end{proof}


\begin{proposition}[Instant-by-instant prediction lower band, MDist-full]
\label{prop.ibib.mdist.lo.full}
Let $\xi_1, \ldots, \xi_n$ be $n$ independent copies of $\xi$.
%
%
%
For any control level $\alpha \in \croch{0, 1}$,
the band $\hat{B}^{\mathrm{MDist-full, lo}}_{\alpha}$ defined as,
for every $\tau \in \brac{1, \ldots, T}$
\begin{align}
\label{eq.ibib.mdist.lo.full}
    \hat{B}_{\alpha; \tau}^{\mathrm{MDist-full, lo}} := \brac{
        k \in \brac{1, \ldots, K} :
        k \geq \hat{Q}_{\tau}^{\mathrm{lo}} \paren{\alpha \paren{1 + \frac{1}{n}} - \frac{1}{n}}
    }
\end{align}
ensures control of the instant-by-instant coverage of $\xi$ at a control level $\alpha$,
where the marginal empirical lower quantile function $\hat{Q}_{\tau}^{\mathrm{lo}}$ is defined in Equation (Eq.~\eqref{eq.eqt.lo.full}). 
\end{proposition}
%
%
%
\begin{proof}[Proof of Proposition \ref{prop.ibib.mdist.lo.full}]
Let $\alpha \in \croch{0, 1}$ and $\tau \in \brac{1, \ldots, T}$.
A full conformal prediction region $\hat{C}_{\alpha; \tau}$ can be defined as
\begin{align*}
    \hat{C}_{\alpha; \tau}
    := \brac{
        k \in \brac{1, \ldots, K}:
        \hat{\pi}\paren{k} > \alpha
    },
\end{align*}
where
\begin{align*}
    \hat{\pi}\paren{k} :=
    \frac{1 
        + \sum_{i=1}^{n}
        \mathbbm{1}
        \brac{
            \hat{F}_{\tau}^{k} \paren{\xi_{i; \tau}}
            \leq \hat{F}_{\tau}^{k} \paren{k}
        }
    }{n+1}.
\end{align*}
%
%
%
Since $\xi_{1; \tau}, \ldots, \xi_{n; \tau}$ are i.i.d. copies of $\xi_{\tau}$,
$\hat{C}_{\alpha; \tau}$ is a confidence prediction region \citep{vovk2005algorithmic}
\begin{align*}
    \mathbb{P} \croch{
        \xi_{\tau} \in \hat{C}_{\alpha; \tau}
    } \geq 1 - \alpha.
\end{align*}
%
%
%
Let us now provide an explicit expression of $\hat{C}_{\alpha; \tau}$.
\begin{align*}
    k \in \hat{C}_{\alpha; \tau}
    &
    \Longleftrightarrow
    \hat{\pi}_{\tau}\paren {k} > \alpha
    \\
    &\Longleftrightarrow
    \frac{1 
        + \sum_{i=1}^{n}
        \mathbbm{1}
        \brac{
            \hat{F}_{\tau}^{k} \paren{\xi_{i; \tau}}
            \leq \hat{F}_{\tau}^{k} \paren{k}
        }
    }{n+1} > \alpha
    \\
    &\Longleftrightarrow
    \frac{1}{n}
    \sum_{i=1}^{n}
    \mathbbm{1}
    \brac{
        \hat{F}_{\tau}^{k} \paren{\xi_{i; \tau}}
        \leq \hat{F}_{\tau}^{k} \paren{k}
    }
    >
    \alpha \paren{1 + \frac{1}{n}} - \frac{1}{n}.
\end{align*}
%
%
%
Let us focus on the l.h.s. term. Let us define the function $\tilde{G}_{\tau} : \brac{1, \ldots, K} \to \croch{0, 1}$
such that for every $k$
\begin{align*}
    \tilde{G}_{\tau}(k)
    &
    :=
    \frac{1}{n}
    \sum_{i=1}^{n}
    \mathbbm{1}
    \brac{
        \hat{F}_{\tau}^{k} \paren{\xi_{i; \tau}}
        \leq \hat{F}_{\tau}^{k} \paren{k}
    }
    \\
    &
    =
    \frac{1}{n}
    \sum_{i=1}^{n}
    \sum_{l=1}^{K}
    \mathbbm{1}
    \brac{l = \xi_{i; \tau}}
    \mathbbm{1}
    \brac{
        \hat{F}_{\tau}^{k} \paren{\xi_{i; \tau}}
        \leq \hat{F}_{\tau}^{k} \paren{k}
    }
    \\
    &
    =
    \sum_{l=1}^{K}
    \hat{p}_{\tau}\paren{l}
    \mathbbm{1}
    \brac{
        \hat{F}_{\tau}^{k} \paren{l}
        \leq \hat{F}_{\tau}^{k} \paren{k}
    }
    \\
    &
    =
    \sum_{l=1}^{K}
    \hat{p}_{\tau}\paren{l}
    \mathbbm{1}
    \brac{
        \frac{n}{n+1}
        \hat{F}_{\tau}\paren{l}
        + \frac{1}{n+1} \mathbbm{1} \brac{k \leq l}
        \leq
        \frac{n}{n+1} 
        \hat{F}_{\tau}\paren{k}
        + \frac{1}{n+1}
    }
    \\
    &
    =
    \sum_{l=1}^{K}
    \hat{p}_{\tau}\paren{l}
    \mathbbm{1}
    \brac{
        \hat{F}_{\tau}\paren{l}
        + \frac{1}{n} \mathbbm{1} \brac{k \leq l}
        \leq \hat{F}_{\tau}\paren{k}
        + \frac{1}{n}
    }
    \\
    &
    =
    \sum_{l=1, l \leq k}^{K}
    \hat{p}_{\tau}\paren{l}
    \mathbbm{1}
    \brac{
        \hat{F}_{\tau}\paren{l}
        \leq \hat{F}_{\tau}\paren{k}
        + \frac{1}{n}
    }
    +
    \sum_{l=1, l > k}^{K}
    \hat{p}_{\tau}\paren{l}
    \mathbbm{1}
    \brac{
        \hat{F}_{\tau}\paren{l}
        \leq \hat{F}_{\tau}\paren{k}
    }.
\end{align*}
If $1 \leq l < k$ then $\hat{F}_{\tau}\paren{l} \leq \hat{F}_{\tau}\paren{k}<\hat{F}_{\tau}\paren{k} + \frac{1}{n}$,
and if $K \geq l > k$ and $\hat{p}_{\tau} \paren{l} \neq 0$, then
\begin{align*}
    \hat{F}_{\tau}\paren{l} = \sum_{\tilde{l}=1}^{l} \hat{p}_{\tau} \paren{\tilde{l}}
    = \sum_{\tilde{l}=1}^{k} \hat{p}_{\tau} \paren{\tilde{l}}
    + \sum_{\tilde{l}=k+1}^{l} \hat{p}_{\tau} \paren{\tilde{l}}
    = \hat{F}_{\tau}\paren{k} + \sum_{\tilde{l}=k+1}^{l} \hat{p}_{\tau} \paren{\tilde{l}}
    > \hat{F}_{\tau}\paren{k},
\end{align*}
since $\sum_{\tilde{l}=k+1}^{l} \hat{p}_{\tau} \paren{\tilde{l}}$ includes $\hat{p}_{\tau} \paren{l} \neq 0$.
%
%
%
Hence, the function $\tilde{G}_{\tau}$ can be rewritten as
\begin{align*}
    \tilde{G}_{\tau} \paren{k} = \sum_{l=1, l \leq k}^{K}
    \hat{p}_{\tau}\paren{l} = \hat{F}_{\tau} \paren{k}.
\end{align*}
%
%
%
Going back to trying to find an expression for $\hat{C}_{\alpha; \tau}$
\begin{align*}
    k \in \hat{C}_{\alpha; \tau}
    &
    \Longleftrightarrow
    \hat{F}_{\tau} \paren{k}
    >
    \alpha \paren{1 + \frac{1}{n}} - \frac{1}{n}
    \\
    &
    \Longleftrightarrow
    k
    \geq
    \tilde{Q}_{\tau} \paren{ \alpha \paren{1 + \frac{1}{n}} - \frac{1}{n}},
\end{align*}
where the empirical marginal lower quantile function is defined as,
for every level $a\in \paren{0, 1}$
\begin{align}
    \label{eq.eqt.lo.full}
    \tilde{Q}_{\tau}^{\mathrm{lo}} \paren{a}
    := \min \brac{
        k \in \brac{1, \ldots, K} :
        \hat{F}_{\tau} \paren{k} > a
    },
\end{align}
the empirical marginal cumulative distribution function $\hat{F}_{\tau}$ in Equation (Eq.~\eqref{eq.ecdf.full}).
\end{proof}

\begin{proposition}[Instant-by-instant prediction band, MDist-full]
\label{prop.ibib.mdist.full}
Let $\xi_1, \ldots, \xi_n$ be $n$ independent copies of $\xi$.
%
%
%
For any control level $\alpha \in \croch{0, 1}$,
and for any $\alpha_{\mathrm{lo}}, \alpha_{\mathrm{up}} \in \croch{0, 1}$ such that $\alpha_{\mathrm{lo}} + \alpha_{\mathrm{up}} = \alpha$,
the band $\hat{B}^{\mathrm{MDist-full}}_{\alpha}$ defined as,
for every $\tau \in \brac{1, \ldots, T}$
\begin{align}
    \label{eq.ibib.mdist.full}
    \hat{B}_{\alpha; \tau}^{\mathrm{MDist-full}}
    :=
    &
    \left\{
    k \in \brac{1, \ldots, K} :
    \hat{Q}_{\tau}^{\mathrm{lo}} \paren{\alpha_{\mathrm{lo}} \paren{1 + \frac{1}{n}} - \frac{1}{n}}
    \leq k,
    \right.
    \notag
    \\
    &
    \qquad
    \qquad
    \qquad
    \qquad
    \left.
    k \leq \hat{Q}_{\tau}^{\mathrm{up}} \paren{\paren{1 - \alpha_{\mathrm{up}}} \paren{1 + \frac{1}{n}}} + 1
    \right\},
\end{align}
ensures control of the instant-by-instant coverage of $\xi$ at a control level $\alpha$.
\end{proposition}
\begin{proof}[Proof of Proposition \ref{prop.ibib.mdist.full}]
Let $\alpha \in \croch{0, 1}$ and $\tau \in \brac{1, \ldots, T}$. It follows from union bound that
\begin{align*}
    \mathbb{P}
    \croch{
        \xi_{\tau}
        \in
        \hat{B}_{\alpha; \tau}^{\mathrm{MDist-full}}
    }
    &
    =
    \mathbb{P}
    \croch{
        \xi_{\tau}
        \in
        \hat{B}_{\alpha_{\mathrm{up}}; \tau}^{\mathrm{MDist-full, up}}
        \cap
        \hat{B}_{\alpha_{\mathrm{lo}}; \tau}^{\mathrm{MDist-full, lo}}
    }
    \\
    &
    =
    1
    -
    \mathbb{P}
    \croch{
        \xi_{\tau}
        \in
        \paren{\hat{B}_{\alpha_{\mathrm{up}}; \tau}^{\mathrm{MDist-full, up}}}^{c}
        \cup
        \paren{\hat{B}_{\alpha_{\mathrm{lo}}; \tau}^{\mathrm{MDist-full, lo}}}^{c}
    }
    \\
    &
    \geq
    1
    -
    \paren{
        \mathbb{P}
    \croch{
        \xi_{\tau}
        \in
        \paren{\hat{B}_{\alpha_{\mathrm{up}}; \tau}^{\mathrm{MDist-full, up}}}^{c}
    }
    +
    \mathbb{P}
    \croch{
        \xi_{\tau}
        \in
        \paren{\hat{B}_{\alpha_{\mathrm{lo}}; \tau}^{\mathrm{MDist-full, lo}}}^{c}
    }
    }
    \\
    &
    \geq
    1
    -
    \paren{ \alpha_{\mathrm{up}} + \alpha_{\mathrm{lo}}
    } = 1 - \alpha.
\end{align*}
\end{proof}


\begin{proposition}[Instant-by-instant prediction band, MDist-Split]
\label{prop.ibib.mdist.split}
Let $\xi_1, \ldots, \xi_n$ be $n$ independent copies of $\xi$.
Let $I_1$ and $I_2$ be two index sets with cardinal $n_1$ and $n_2$ respectively such that
$I_1 \sqcup I_2 = \brac{1, \ldots, n}$.
%
%
%
For any control level $\alpha \in \croch{0, 1}$,
the band $\hat{B}^{\mathrm{MDist-split}}_{\alpha}$ defined as,
for every $\tau \in \brac{1, \ldots, T}$
\begin{align}
    \label{eq.ibib.mdist.split}
    \hat{B}_{\alpha; \tau}^{\mathrm{MDist-split}} :=
    \brac{
        k \in \brac{1, \ldots, K} :
        \tilde{Q}_{D_1; \tau}^{\mathrm{lo}} \paren{\hat{t}_{\alpha; \tau}^{\mathrm{MDist-split}}}
        \leq k \leq \tilde{Q}_{D_1; \tau}^{\mathrm{up}} \paren{1 - \hat{t}_{\alpha; \tau}^{\mathrm{MDist-split}}}
    },
\end{align}
ensures control of the instant-by-instant coverage of $\xi$ at a control level $\alpha$,
where the threshold $\hat{t}_{\alpha; \tau}^{\mathrm{MDist-split}}$ is defined in Equation (Eq.~\eqref{eq.emp.conf.level})
and the marginal empirical upper and lower quantile function $\tilde{Q}_{D_1; \tau}^{\mathrm{up}}$
and $\tilde{Q}_{D_1; \tau}^{\mathrm{lo}}$ are defined in Equations (Eq.~\eqref{eq.eqt.up.split}) and (Eq.~\eqref{eq.eqt.lo.split})
respectively.
\end{proposition}
%
%
%
\begin{proof}[Proof of Proposition \ref{prop.ibib.mdist.split}]
Let $\tau \in \brac{1, \ldots, T}$.
Define the empirical marginal cumulative density function $\hat{F}_{D_1; \tau}$ trained on the data set
$D_{1} := \brac{\xi_{i}, i \in I_1}$ as, for every $k \in \brac{1, \ldots, K}$
\begin{align}
\label{eq.ecdf.D1}
\hat{F}_{D_1; \tau} \paren{k} := \frac{1}{n_1} \sum_{i \in I_1} \mathbbm{1} \brac{\xi_{i; \tau} \leq k}.
\end{align}
%
%
%
Upon the dataset $D_{2} := \brac{\xi_{i}, i \in I_2}$,
one can define split conformal prediction region $\hat{C}_{\alpha; \tau}$ as
\begin{align*}
    \hat{C}_{\alpha; \tau}
    := \brac{
        k \in \brac{1, \ldots, K}
        : \hat{\pi}_{D_2, \tau} \paren{k} > \alpha
    },
\end{align*}
where for every $k \in \paren{1, \ldots, K}$
\begin{align*}
    \hat{\pi}_{D_2, \tau} \paren{k}
    := \frac{
        1 + \sum_{i \in \mathrm{I_2}}
        \mathbbm{1}
        \brac{
            \hat{A}_{D_1; \tau} \paren{\xi_{i; \tau}}
            \leq
            \hat{A}_{D_1; \tau} \paren{k}
        }
    }{1+ n_2},
\end{align*}
where the conformity score $\hat{A}_{D_1; \tau} \paren{k} := \min \paren{\hat{F}_{D_1; \tau} \paren{k}, 1 - \hat{F}_{D_1; \tau} \paren{k}}$.
%
%
%
Since $\xi_{1; \tau}, \ldots, \xi_{n; \tau}$ are i.i.d. copies of $\xi_{\tau}$,
it follows that $\hat{C}_{\alpha; \tau}$ is a confidence prediction region
\begin{align*}
    \mathbb{P}
    \paren{
        \xi_{\tau} \in \hat{C}_{\alpha; \tau}
    } \geq 1 - \alpha.
\end{align*}
%
%
%
Let us now compute an explicit expression for $\hat{C}_{\alpha; \tau}$.
\begin{align*}
    &k \in \hat{C}_{\alpha; \tau}
    \\
    \Longleftrightarrow
    &\frac{
        1 + \sum_{i \in \mathrm{I_2}}
        \mathbbm{1}
        \brac{
            \hat{A}_{D_1; \tau} \paren{\xi_{i; \tau}}
            \leq
            \hat{A}_{D_1; \tau} \paren{k}
        }
    }{1+ n_2} > \alpha
    \\
    \Longleftrightarrow
    &\frac{
        1
    }{n_2}
    \sum_{i \in \mathrm{I_2}}
    \mathbbm{1}
    \brac{
        \hat{A}_{D_1; \tau} \paren{\xi_{i; \tau}}
        \leq
        \hat{A}_{D_1; \tau} \paren{k}
    } > \alpha \paren{1 + \frac{1}{n_2}} - \frac{1}{n_2}
    \\
    \Longleftrightarrow
    &
    \hat{A}_{D_1; \tau} \paren{k}
    \geq \hat{\ell}_{\alpha; \tau},
\end{align*}
where the level $\hat{t}_{\alpha; \tau}^{\mathrm{MDist-split}}$ is defined as
\begin{align}
\label{eq.emp.conf.level}
    \hat{t}_{\alpha; \tau}^{\mathrm{MDist-split}}
    &
    := \inf
    \brac{
        t \in \croch{0, 1} :
        \frac{
            1
        }{n_2}
        \sum_{i \in \mathrm{I_2}}
        \mathbbm{1}
        \brac{
            \hat{A}_{D_1; \tau} \paren{\xi_{i; \tau}}
            \leq t
        } > \alpha \paren{1 + \frac{1}{n_2}} - \frac{1}{n_2}
    }
    \notag
    \\
    &
    =
    \hat{A}_{D_1; \tau} \paren{\xi_{\paren{i_{n_2; \alpha}}, \tau}},
\end{align}
where the index  $i_{n_2; \alpha} := \ceil{\alpha \paren{n_2 + 1} - 1}$,
and $\hat{A}_{D_1; \tau} \paren{\xi_{\paren{1}; \tau}} \leq \ldots \leq \hat{A}_{D_1; \tau} \paren{\xi_{\paren{n_2}; \tau}}$
are the conformity scores sorted in increasing order.
%
%
%
Going back to trying to find an expression of the prediction region
\begin{align*}
    &k \in \hat{C}_{\alpha; \tau}
    \\
    \Longleftrightarrow
    &
    \min \paren{
        \hat{F}_{D_1; \tau} \paren{k},
        1 - \hat{F}_{D_1; \tau} \paren{k} 
    }
    \geq \hat{t}_{\alpha; \tau}^{\mathrm{MDist-split}},
    \\
    \Longleftrightarrow
    &
    \hat{F}_{D_1; \tau} \paren{k}
    \geq \hat{t}_{\alpha; \tau}^{\mathrm{MDist-split}},
    \mbox{ and }
    1 - \hat{F}_{D_1; \tau} \paren{k} 
    \geq \hat{t}_{\alpha; \tau}^{\mathrm{MDist-split}},
    \\
    \Longleftrightarrow
    &
    \hat{t}_{\alpha; \tau}^{\mathrm{MDist-split}}
    \leq \hat{F}_{D_1; \tau} \paren{k}
    \leq 1 - \hat{t}_{\alpha; \tau}^{\mathrm{MDist-split}},
    \\
    \Longleftrightarrow
    &
    \tilde{Q}_{D_1; \tau}^{\mathrm{lo}} \paren{\hat{t}_{\alpha; \tau}^{\mathrm{MDist-split}}}
    \leq k
    \leq \tilde{Q}_{D_1; \tau}^{\mathrm{up}} \paren{1 - \hat{t}_{\alpha; \tau}^{\mathrm{MDist-split}}},
\end{align*}
where the empirical marginal upper quantile function built from $D_1$ is defined as,
for every level $a \in \croch{0, 1}$ is defined as 
\begin{align}
\label{eq.eqt.up.split}
    \tilde{Q}_{D_1; \tau}^{\mathrm{up}} \paren{1 - a}
    := \max \brac{
        k \in \brac{1, \ldots, K} :
        \hat{F}_{D_1; \tau} \paren{k} \leq 1 - a
    },
\end{align}
as for the lower one
\begin{align}
\label{eq.eqt.lo.split}
    \tilde{Q}_{D_1; \tau}^{\mathrm{lo}} \paren{a}
    := \min \brac{
        k \in \brac{1, \ldots, K} :
        \hat{F}_{D_1; \tau} \paren{k} \geq a
    }.
\end{align}
\end{proof}

\begin{proposition}[$\gamma$-simultaneous band]
\label{prop.make.it.sim}
Let $\xi_1, \ldots, \xi_{n}$ be i.i.d. copies of $\xi$.
Let $I_1$ and $I_2$ be two index sets with cardinal $n_1$ and $n_2$ respectively
such that $I_1 \sqcup I_2 = \brac{1, \ldots, n}$.
%
%
%
Let $\hat{A}_{D_1; \tau} : \brac{1, \ldots, K} \to \mathbb{R}$, $k \mapsto \hat{A}_{D_1; \tau} \paren{k}$
be a conformity-measure over $\brac{1, \ldots, K}$ built from the dataset $D_1 := \brac{\xi_{i} : i \in I_1}$.
%
%
%

For any slack level $\gamma \in \croch{0, 1}$, 
control level $\alpha \in \croch{0, 1}$,
the prediction band $\hat{B}_{\gamma; \alpha}$ defined as for every $\tau \in \brac{1, \ldots, T}$
\begin{align*}
    \hat{B}_{\gamma; \alpha; \tau}
    &
    := \brac{
        k \in \brac{1, \ldots, K} : \hat{A}_{D_1; \tau} \paren{k} \geq \hat{t}_{D_1; \gamma} \paren{\xi_{\paren{i_{n_2; \alpha}}}}
    }
\end{align*}
ensures control of the $\gamma$-simultaneous coverage of $\xi$ at a level $\alpha$
where the cut-off level $\hat{t}_{D_1; \gamma} \paren{\xi_{\paren{i_{n_2; \alpha}}}}$
is defined in Equation (Eq.~\ref{eq.sim.cutoff}) from the conformity score defined in Equation (Eq.~\ref{eq.sim.conformity.score}).
\end{proposition}
%
%
%
\begin{proof}[Proof of Proposition \ref{prop.make.it.sim}]
Let $\gamma \in \croch{0, 1}$ and $\alpha \in \croch{0, 1}$.
One can define a conformity measure $\hat{t}_{D_1; \gamma} : \brac{1, \ldots, K}^{T} \to \mathbb{R}$ such that
for every $\xi \in \brac{1, \ldots, K}^{T}$
\begin{align}
\label{eq.sim.conformity.score}    
    \hat{t}_{D_1; \gamma} \paren{\xi}
    := \sup
    \brac{
        t \in \mathbb{R}
        :
        \frac{1}{T}
        \sum_{\tau = 1}^{T}
        \mathbbm{1} \brac{
            \hat{A}_{D_1; \tau} \paren{\xi_{\tau}} \geq t
        }
        \geq 1 - \gamma
    }
    = \hat{A}_{D_1; \paren{\tau_{\gamma, T}}} \paren{\xi_{\paren{\tau_{\gamma, T}}}},
\end{align}
where $\hat{A}_{D_1; \paren{1}} \paren{\xi_{\paren{1}}} \geq \ldots \geq \hat{A}_{D_1; \paren{T}} \paren{\xi_{\paren{T}}}$
are sorted in decreasing order and $\tau_{\gamma, T} := \ceil{T\paren{1 - \gamma}}$.
%
%
%
Upon the dataset $D_2 := \brac{\xi_{i} : i \in I_2}$,
one can define a split conformal prediction region as
\begin{align*}
    \hat{C}_{\gamma; \alpha}
    := \brac{
        \xi \in \brac{1, \ldots, K}^{T} : \hat{\pi}_{D_2} \paren{\xi} > \alpha
    },
\end{align*}
where for every $\xi \in \brac{1, \ldots, K}^{T}$
\begin{align*}
    \hat{\pi}_{D_2} \paren{\xi}
    := \frac{
        1
        + \sum_{i \in I_2}
        \brac{\hat{t}_{D_1; \gamma} \paren{\xi_{i}} \leq \hat{t}_{D_1; \gamma} \paren{\xi}}
    }{n_2+1}.
\end{align*}
%
%
%
Let us now provide an explicit expression for $\hat{C}_{\gamma; \alpha}$,
\begin{align}
\label{eq.sim.cutoff}
    \xi \in \hat{C}_{\gamma; \alpha}
    &
    \Longleftrightarrow
    \frac{
        1
        + \sum_{i \in I_2}
        \brac{\hat{t}_{D_1; \gamma} \paren{\xi_{i}} \leq \hat{t}_{D_1; \gamma} \paren{\xi}}
    }{n_2+1} > \alpha
    \notag
    \\
    &
    \Longleftrightarrow
    \frac{1}{n_2}
    \sum_{i \in I_2}
    \brac{\hat{t}_{D_1; \gamma} \paren{\xi_{i}} \leq \hat{t}_{D_1; \gamma} \paren{\xi}}
    > \paren{1 + \frac{1}{n_2}}\alpha - \frac{1}{n_2}
    \notag
    \\
    &
    \Longleftrightarrow
    \hat{t}_{D_1; \gamma} \paren{\xi}
    \geq \hat{t}_{D_1; \gamma} \paren{\xi_{\paren{i_{n_2; \alpha}}}},
\end{align}
where $\hat{t}_{D_1; \gamma} \paren{\xi_{\paren{i_{n_2; \alpha}}}} \leq \ldots \leq \hat{t}_{D_1; \gamma} \paren{\xi_{\paren{i_{n_2; \alpha}}}}$
are sorted in increasing order and $i_{n_2; \alpha} :=$ $\ceil{\paren{n_2 + 1}\alpha - 1}$.
%
%
%
Hence, the prediction band $\hat{B}_{\gamma; \alpha}$ defined as for every $\tau \in \brac{1, \ldots, T}$
\begin{align*}
    \hat{B}_{\gamma; \alpha; \tau}
    &
    := \brac{
        k \in \brac{1, \ldots, K} : \hat{A}_{D_1; \tau} \paren{k} \geq \hat{t}_{D_1; \gamma} \paren{\xi_{\paren{i_{n_2; \alpha}}}}
    }
\end{align*}
ensures the following
%
%
%
\begin{align*}
    \mathbb{P}
    \croch{
        \frac{1}{T}
        \sum_{\tau=1}^{T}
        \mathbbm{1}
        \brac{
            \xi_{\tau} \in \hat{B}_{\gamma; \alpha; \tau}
        }
    }
    &
    =
    \mathbb{P}
    \croch{
        \frac{1}{T}
        \sum_{\tau=1}^{T}
        \mathbbm{1}
        \brac{
            \hat{A}_{D_1; \tau}\paren{\xi_{\tau}} \geq \hat{t}_{D_1; \gamma} \paren{\xi_{\paren{i_{n_2; \alpha}}}}
        }
    }
    \\
    &
    =
    \mathbb{P}
    \croch{
        \hat{t}_{D_1; \gamma} \paren{\xi} \geq \hat{t}_{D_1; \gamma} \paren{\xi_{\paren{i_{n_2; \alpha}}}}
    }
    \\
    &
    =
    \mathbb{P}
    \croch{
        \xi \in \hat{C}_{\gamma, \alpha}
    } \geq 1 - \alpha.
\end{align*}
\end{proof}

%
%
%

\begin{corollary}[$\gamma$-simultaneous band, MD-Split]
\label{cor.sim.md.split}
Let $\xi_1, \ldots, \xi_{n}$ be i.i.d. copies of $\xi$.
Let $I_1$ and $I_2$ be two index sets with cardinal $n_1$ and $n_2$ respectively
such that $I_1 \sqcup I_2 = \brac{1, \ldots, n}$.

%
%
%

For any slack $\gamma \in (0,1)$, control level $\alpha \in \croch{0, 1}$,
the prediction band $\hat{B}^{\mathrm{MD-split}}_{\gamma; \alpha}$ such that for every $\tau \in \brac{1, \ldots, T}$
\begin{align}
    \label{eq.sim.md.split}
    \hat{B}^{\mathrm{MD-split}}_{\gamma; \alpha; \tau} :=
    \brac{
        k \in \brac{1, \ldots, K} :
        \hat{p}_{D_1; \tau}\paren{k}
        \geq
        \hat{\ell}_{\gamma; \alpha}^{\mathrm{MD-Split}}
    },
\end{align}
ensures control of $\gamma$-simultaneous coverage of $\xi$ at a control level $\alpha$,
where empirical marginal probability density function $\hat{p}_{D_1; \tau}$ is defined in Equation (Eq.~\eqref{eq.epmf.D1}),
and the threshold is  $\hat{\ell}_{\gamma; \alpha}^{\mathrm{MD-Split}}$ is defined in Equation (Eq.~\eqref{eq.threshold.sim.md.split}).
\end{corollary}
%
%
%
\begin{proof}[Proof of Corollary \ref{cor.sim.md.split}]
For every $\tau \in \brac{1, \ldots, K}$, define
empirical marginal probability density function $\hat{p}_{D_1; \tau}$ built from
the dataset $D_{1} := \brac{\xi_i : i \in I_1}$ as, for every $k \in \brac{1, \ldots, K}$
\begin{align}
\label{eq.epmf.D1}
    \hat{p}_{D_1; \tau}\paren{k}
    := \frac{1}{n_1} \sum_{i \in I_1} \mathbbm{1} \brac{\xi_{i; \tau} = k}.
\end{align}
%
%
%
Applying Proposition \ref{prop.make.it.sim} by choosing for every $\tau \in \brac{1, \ldots, T}$,
and every $k \in \brac{1, \ldots, K}$, $\hat{A}_{D_1; \tau} \paren{k} := \hat{p}_{D_1; \tau} \paren{k}$,
the prediction band $\hat{B}^{\mathrm{MD-split}}_{\gamma; \alpha}$ defined as,
for every $\tau \in \brac{1, \ldots, T}$
\begin{align*}
    \hat{B}^{\mathrm{MD-split}}_{\gamma; \alpha; \tau}
    &
    :=
    \brac{
        k \in \brac{1, \ldots, K} :
        \hat{A}_{D_1; \tau} \paren{k}
        \geq
        \hat{t}_{D_1; \gamma} \paren{\xi_{\paren{i_{n_2; \alpha}}}}
    }
    \\
    &
    = \brac{
        k \in \brac{1, \ldots, K} :
        \hat{p}_{D_1; \tau} \paren{k}
        \geq
        \hat{\ell}_{\gamma; \alpha}^{\mathrm{MD-Split}}
    },
\end{align*}
ensures control of $\gamma$-simultaneous coverage of $\xi$ at a control level $\alpha$,
where the threshold $\hat{\ell}_{\gamma; \alpha}^{\mathrm{MD-Split}}$ is defined as
\begin{align}
    \label{eq.threshold.sim.md.split}
    \hat{\ell}_{\gamma; \alpha}^{\mathrm{MD-Split}}
    := \hat{t}_{D_1; \gamma} \paren{\xi_{\paren{i_{n_2; \alpha}}}}.
\end{align}
\end{proof}

%
%
%

\begin{corollary}[$\gamma$-simultaneous band, MHPD-Split]
\label{cor.sim.mhpd.split}
Let $\xi_1, \ldots, \xi_{n}$ be i.i.d. copies of $\xi$.
Let $I_1$ and $I_2$ be two index sets with cardinal $n_1$ and $n_2$ respectively
such that $I_1 \sqcup I_2 = \brac{1, \ldots, n}$.

%
%
%

For any slack $\gamma \in (0,1)$, control level $\alpha \in \croch{0, 1}$,
the prediction band $\hat{B}^{\mathrm{MHPD-split}}_{\gamma; \alpha}$ such that for every $\tau \in \brac{1, \ldots, T}$
\begin{align}
    \label{eq.sim.mhpd.split}
    \hat{B}_{\gamma; \alpha; \tau}^{\mathrm{MHPD-split}} :=
    \brac{
        k \in \brac{1, \ldots, K} :
        \hat{p}_{D_1; \tau} \paren{k} > \hat{\ell}_{\gamma; \alpha; \tau}^{\mathrm{MHPD-split}}
    },
\end{align}
ensures $\gamma$-simultaneous coverage of $\xi$ at a control level $\alpha$,
where empirical marginal probability density function $\hat{p}_{D_1; \tau}$ is defined in Equation (Eq.~\eqref{eq.epmf.D1}),
and the threshold $\hat{\ell}_{\gamma; \alpha; \tau}^{\mathrm{MHPD-split}}$ is defined in Equation (Eq.~\eqref{eq.threshold.sim.hpd.split}).
\end{corollary}
%
%
%
\begin{proof}[Proof of Corollary \ref{cor.sim.mhpd.split}]
For every $\tau \in \brac{1, \ldots, K}$, and every $k \in \brac{1, \ldots, K}$
define the rank $\hat{R}_{D_1; \tau}\paren{k}$ of $k$ at the instant $\tau$
built from the dataset $D_1 := \brac{\xi_i : i \in I_1}$ as
\begin{align}
\label{eq.rank.D1}
    \hat{R}_{D_1; \tau}\paren{k}
    :=
    \sum_{l = 1}^{K}
    \hat{p}_{D_1; \tau} \paren{l}
    \mathbbm{1} \brac{
        \hat{p}_{D_1; \tau} \paren{l}
        \leq \hat{p}_{D_1; \tau} \paren{k}
    }.
\end{align}
%
%
%
Applying Proposition \ref{prop.make.it.sim} by choosing for every $\tau \in \brac{1, \ldots, T}$,
and every $k \in \brac{1, \ldots, K}$, $\hat{A}_{D_1; \tau} \paren{k} := \hat{R}_{D_1; \tau} \paren{k}$,
the prediction band $\hat{B}^{\mathrm{MD-split}}_{\gamma; \alpha}$ defined as,
for every $\tau \in \brac{1, \ldots, T}$
\begin{align*}
    \hat{B}_{\gamma; \alpha; \tau}^{\mathrm{MHPD-split}}
    &
    :=
    \brac{
        k \in \brac{1, \ldots, K} :
        \hat{A}_{D_1; \tau} \paren{k}
        \geq
        \hat{t}_{D_1; \gamma} \paren{\xi_{\paren{i_{n_2; \alpha}}}}
    }
    \\
    &
    = \brac{
        k \in \brac{1, \ldots, K} :
        \hat{R}_{D_1; \tau} \paren{k}
        \geq
        \hat{t}_{D_1; \gamma} \paren{\xi_{\paren{i_{n_2; \alpha}}}}
    }
    \\
    &
    =
    \brac{
        k \in \brac{1, \ldots, K} :
        \hat{p}_{D_1; \tau}\paren{k} > \hat{\ell}_{\gamma; \alpha; \tau}^{\mathrm{MHPD-split}}
    },
\end{align*}
ensures control of $\gamma$-simultaneous coverage of $\xi$ at a control level $\alpha$,
where the cut-off $\hat{\ell}_{\gamma; \alpha; \tau}^{\mathrm{MHPD-split}}$ is defined as
\begin{align}
\label{eq.threshold.sim.hpd.split}
    \hat{\ell}_{\gamma; \alpha; \tau}^{\mathrm{MHPD-split}}
    := \sup
    \brac{
        \ell \in \croch{0, 1}
        : \sum_{l = 1}^{K}
        \hat{p}_{D_1; \tau} \paren{l}
        \mathbbm{1} \brac{
            \hat{p}_{D_1; \tau} \paren{l}
            > \ell
        } 
        > 1 - \hat{t}_{D_1; \gamma}^{\mathrm{MHPD-split}} \paren{\xi_{\paren{i_{n_2; \alpha}}}}
    },
\end{align}   
\end{proof}


\begin{corollary}[$\gamma$-simultaneous band, MDist-Split]
\label{cor.sim.mdist.split}
Let $\xi_1, \ldots, \xi_n$ be $n$ independent copies of $\xi$.
Let $I_1$ and $I_2$ be two index sets with cardinal $n_1$ and $n_2$ respectively such that
$I_1 \sqcup I_2 = \brac{1, \ldots, n}$.


For any slack $\gamma \in (0,1)$, any control level $\alpha \in \croch{0, 1}$,
the band $\hat{B}^{\mathrm{MDist-split}}_{\gamma; \alpha}$ defined as,
for every $\tau \in \brac{1, \ldots, T}$
\begin{align}
    \label{eq.sim.mdist.split}
    \hat{B}_{\gamma; \alpha; \tau}^{\mathrm{MDist-split}} :=
    \brac{
        k \in \brac{1, \ldots, K} :
        \hat{Q}_{D_1, \tau}^{\mathrm{lo}}
        \paren{
            \hat{t}_{\gamma; \alpha}^{\mathrm{MDist-split}}
        }
        \leq k
        \leq \hat{Q}_{D_1, \tau}^{\mathrm{lo}}
        \paren{
            1 - \hat{t}_{\gamma; \alpha}^{\mathrm{MDist-split}}
        }
    },
\end{align}
ensures $\gamma$-simultaneous coverage of $\xi$ at a control level $\alpha$,
where the level $\hat{t}_{\gamma; \alpha}^{\mathrm{MDist-split}}$ is defined in Equation (Eq.~\eqref{eq.threshold.sim.mdist.split}).
\end{corollary}
%
%
%
\begin{proof}[Proof of Corollary \ref{cor.sim.mdist.split}]
Applying Proposition \ref{prop.make.it.sim},
for every $\tau \in \brac{1, \ldots, K}$, and $k \in \brac{1, \ldots, K}$
\begin{align}
\label{eq.sim.ecdf}
    \hat{A}_{D_1; \tau}\paren{k}
    := \min\paren{\hat{F}_{D_1; \tau} \paren{k}, 1 - \hat{F}_{D_1; \tau} \paren{k}},
\end{align}
%
%
%
the prediction band $\hat{B}^{\mathrm{MDist-split}}_{\gamma; \alpha}$ defined as,
for every $\tau \in \brac{1, \ldots, T}$
\begin{align*}
    \hat{B}_{\gamma; \alpha; \tau}^{\mathrm{MHPD-split}}
    &
    :=
    \brac{
        k \in \brac{1, \ldots, K} :
        \hat{A}_{D_1; \tau} \paren{k}
        \geq
        \hat{t}_{D_1; \gamma} \paren{\xi_{\paren{i_{n_2; \alpha}}}}
    }
    \\
    &
    = \brac{
        k \in \brac{1, \ldots, K} :
        \min \paren{
            \hat{F}_{D_1; \tau} \paren{k},
            1 - \hat{F}_{D_1; \tau} \paren{k}   
        }
        \geq \hat{t}_{\gamma; \alpha}^{\mathrm{MDist-split}}
    }
    \\
    &
    =
    \brac{
        k \in \brac{1, \ldots, K} :
        \hat{Q}_{D_1, \tau}^{\mathrm{lo}}
        \paren{
            \hat{t}_{\gamma; \alpha}^{\mathrm{MDist-split}}
        }
        \leq k
        \leq \hat{Q}_{D_1, \tau}^{\mathrm{lo}}
        \paren{
            1 - \hat{t}_{\gamma; \alpha}^{\mathrm{MDist-split}}
        }
    },
\end{align*}
ensures control of $\gamma$-simultaneous coverage of $\xi$ at a control level $\alpha$,
where the level $\hat{t}_{\gamma; \alpha}^{\mathrm{MDist-split}}$ is defined as
\begin{align}
    \label{eq.threshold.sim.mdist.split}
    \hat{t}_{\gamma; \alpha}^{\mathrm{MDist-split}} :=
    \hat{t}_{D_1; \gamma} \paren{\xi_{\paren{i_{n_2; \alpha}}}}.
\end{align}
\end{proof}


% \begin{proposition}[Make it high probability.]
% Let $\xi_1, \ldots, \xi_n$ be $n$ independent copies of $\xi$.
% Let $I_1$ and $I_2$ be two index sets with cardinal $n_1$ and $n_2$ respectively such that
% $I_1 \sqcup I_2 = \brac{1, \ldots, n}$.

% For any risk level $\delta$ and a control level $\delta$,
% \begin{align*}
%     \mathbb{P}
%     \croch{
%         \mathbb{P}
%         \croch{
%             \xi \in \hat{C}_{a_{n_2; \alpha; \delta}} \mid D_2
%         } \geq 1 - \alpha
%     } \geq 1 - \delta,
% \end{align*}
% where the level $a_{n_2; \alpha; \delta}$ is given by
% \begin{align*}
%     a_{n_2; \alpha; \delta}
% \end{align*}

% \end{proposition}
% \begin{proof}
% \begin{align*}
%     \mathbb{P}
%     \croch{
%         \mathbb{P}
%         \croch{
%             \mid D_2
%         } \geq 1 - \alpha
%     } \geq 1 - \delta. 
% \end{align*}
% \end{proof}

\printbibliography
%
%
%



\end{document}


